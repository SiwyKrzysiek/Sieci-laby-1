\documentclass{article}

\usepackage{polski} % Pozwala na użycie polskiego. Ustawia między innymi fontenc na T1
\usepackage[utf8]{inputenc} % Informuje o kodowaniu

\usepackage{graphicx}
\graphicspath{ {./Obrazy/} }

\usepackage{textcomp} % Znaki specjalne takie jak ~

\title{Laboratorium sieci komputerowych - ćw. 1 \\ Podstawy pracy na systemie z rodziny Linux}
\author{Krzysztof Dąbrowski gr. 3}
\date{\today}

\begin{document}
\maketitle{}
\tableofcontents{}
\newpage

\section{Wstęp}
Celem zajęć laboratoryjnych było zaznajomienie się z podstawami pracy i administracji systemów z rodziny Linux. Przed przystąpieniem do pracy należało opanować podstawowe pojęcia związanie z kryptografią asymetryczną oraz poznać wybrane programy narzędziowe zainstalowane na komputerach laboratoryjnych.

\section{Kryptografia asymetryczna}
Jest to część kryptografii, gdzie operacja szyfrowania i odszyfrowywania informacji znacząco się od siebie różnią.

\subsection{Sposób działania}
Algorytmy stosowane w kryptografii asymetrycznej pozwalają na wygenerowanie pary powiązanych ze sobą kluczy. Jedne z nich jedynie na szyfrowanie danych, a drugi na ich odszyfrowanie. Klucz szyfrujący jest też nazywany \textit{publicznym} a odszyfrowujący \textit{prywatnym}.

Dzięki zastosowaniu kluczny asymetrycznych możliwe jest bezpieczne przesłanie wiadomości miedzy stronami chcącymi się porozumieć.

\begin{figure}[h]
    \centering
    \includegraphics[width=8cm]{schemat_szyfrowania_asymetrycznego}
    \caption{Schemat transmisji danych}
\end{figure}

\subsection{Zastosowanie}
Dzięki wykorzystaniu kryptografii asymetrycznej możliwe jest nawiązanie połączenia z serwerem bez potrzeby przesyłania hasła. Przeprowadzenie uwierzytelnienia w ten sposób polega na zaszyfrowaniu przez serwer danych kluczem publicznym osoby chcącej się zalogować. Następnie serwer oczekuje na odesłanie odszyfrowanej danej. Ponieważ odszyfrowanie jest możliwe tylko przez posiadacza klucza prywatnego, to serwer jest w ten sposób w stanie potwierdzić autentyczność osoby logującej się.

Należy zwrócić uwagę, żę podczas takiego procesu nie są przesyłane żadne wrażliwe dane. Jest on więc bezpieczny nawet w przypadku podsłuchania. W przeciwieństwie to tradycyjnego logowania.

\subsection{Generowanie kluczy}
Połączoną parę kluczy można wygenerować przy pomocy polecenia \texttt{ssh-keygen}. Przy domyślnch ustawieniach zostaną one zapisanie w katalogu \texttt{{\raise.17ex\hbox{$\scriptstyle\sim$}}/.ssh}.

\subsection{Przekazanie klucza}
Jedyną pozostałą czynnością jest przekazanie klucza publicznego do serwera, do którego planuje się logować. Można to łatwo osiągnąć przy pomocy polecenia \texttt{ssh-copy-id}.

Po wykonaniu tych czynności możliwe jest logowanie przez \texttt{ssh} bez podawania hasła.

\section{Zarządzanie uprawnieniami administratorskimi}
W systemach zazwyczaj nie pracuje się na koncie administratorskim. Zamiast tego powinno się korzystać ze standardowego konta. Gdy potrzebne są specjalne uprawnienia należy wykonać dane polecenie z użytkownikiem zmienionym na administratora (root).

\subsection{Zmiana użytkownika}
Do zmiany użytkownika wykorzystać można polecenie \texttt{su}.

Najczęściej jednak zachodzi potrzeba wykonania pojedynczego polecenia jako root. Służy do tego polecenie \texttt{sudo}. W przeciwieństwie do trwałej zmiany użytkownika po wywołaniu \texttt{sudo} użytkownik zostanie poproszony o podanie hasła do konta, z którego aktualnie korzysta, a nie z tego, jako które chce wykonać polecenie.

\subsection{Dostęp do sudo}
Decyzję o tym czy użytkownik może korzystać z polecenia \texttt{sudo} system podejmuje na podstawie informacji zawartych w pliku \texttt{/etc/sudoers}. Zawiera on listę grup i użytkowników, którzy mają taką możliwość.

Plik ten może posiadać nawiązania do wybranych katalogów. W takiej sytuacji do podjęcia decyzji o dostępie do \texttt{sudo} brane są pod uwagę również wszystkie pliki znajdujące się we wskazanym katalogu.

\subsection{Zablokowanie sudo wszystkim innym}
W celu zostania jedynym użytkownikiem mającym dostęp do \texttt{sudo} najłatwiej jest zakomentować wszystkie linie dający uprawnianie grupom i użytkownikom oraz dodać uprawniania sobie edytując plik \texttt{/etc/sudoers}. Należy również koniecznie postawić drugi znak \texttt{\#} przed liniami linków do innych plików.

\subsection{Edycja pliku sudoers}
Ponieważ plik \texttt{/etc/sudoers} jest kluczowy do poprawnego działania systemu do jego edycje zalecane jest użycie programu \texttt{visudo}. Otwiera on domyślnie plik \texttt{/etc/sudoers} przy pomocy domyślnego edytora a po zakończeniu edycji sprawdza poprawność powstałego pliku.

Możliwe jest wybranie edytora używanego przez \texttt{visudo} poprzez ustawienie zmiennej \texttt{EDITOR}. \par{}
Ponieważ zazwyczaj do edycji pliku \texttt{sudoers} wymagane są uprawniania administratorskie to przykładowe wywołanie z wybraniem edytora może wyglądać następująco: \texttt{sudo EDITOR=vim visudo}.

Z powodów bezpieczeństwa domyślnie przy użyciu \texttt{sudo} czyszczone są zmienne środowiskowe. By skorzystać z wartości globalnej zmiennej \texttt{EDITOR} należy wywołać \texttt{sudo}  z flagą \texttt{-E}.

\section{Blokowanie zdalnego dostępu do komputera}
Za możliwość zdalnego logowania na stację roboczą odpowiada usługa \texttt{sshd}.

By dowiedzieć się czy ta usługa jest aktualnie włączona można skorzystać z połączenia poleceń \texttt{ps -A | grep ssh}. \par{}
\texttt{ps -A} -- Wyświetla listę wszystkich działających procesów. \par{}
\texttt{grep ssh} -- Wybiera te elementy, które zawierają tekst ,,ssh''.

\subsection{Zmiana stanu procesu}
By zatrzymać proces związany z ssh można użyć polecenia \texttt{systemctl stop ssh}. Po jego wykonaniu nie będzie możliwe zdalne logowanie.

Proces można uruchomić ponownie przy pomocy polecenia \texttt{systemctl start ssh}.



\end{document}