\documentclass{article}

\usepackage{polski} % Pozwala na użycie polskiego. Ustawia między innymi fontenc na T1
\usepackage[utf8]{inputenc} % Informuje o kodowaniu

\title{Laboratorium sieci komputerowych - ćw. 1 \\ Podstawy pracy na systemie z rodziny Linux}
\author{Krzysztof Dąbrowski gr. 3}
\date{\today}

\begin{document}
\maketitle{}
\tableofcontents{}
\newpage

\section{Wstęp}
Celem zajęć laboratoryjnych było zaznajomienie się z podstawami pracy i administracji systemów z rodziny Linux. Przed przystąpieniem do pracy należało opanować podstawowe pojęcia związanie z kryptografią asymetryczną oraz poznać wybrane programy narzędziowe zainstalowane na komputerach laboratoryjnych.

\section{Kryptografia asymetryczna}
Jest to część kryptografii, gdzie operacja szyfrowania i odszyfrowywania informacji są od siebie istotnie różne.

\section{Zarządzanie uprawnieniami administratorskimi}

\section{Blokowanie zdalnego dostępu do komputera}


\end{document}
\maketitle{}
\tableofcontents{}
